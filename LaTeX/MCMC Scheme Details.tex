\documentclass[12pt,a4paper]{article}
\usepackage[margin=1in]{geometry}
\usepackage[round]{natbib}
\usepackage{amsfonts, amsthm, amsmath, amssymb}
\usepackage{mathtools}
\usepackage{cancel}
\usepackage{enumerate}
\usepackage{bbm}
\usepackage{caption}
\usepackage{xcolor, soul}
\usepackage{subcaption}
\usepackage{booktabs, multirow}
\usepackage{comment}
\usepackage[most]{tcolorbox}


%%%%% Clickable link navigation
\usepackage{hyperref}
\hypersetup{colorlinks,
	citecolor=black,
	filecolor=black,
	linkcolor=black,
	urlcolor=black
}

\newcommand{\mathcolorbox}[2]{\colorbox{#1}{$\displaystyle #2$}}

%%%%% Page style and header
\usepackage{fancyhdr}
\pagestyle{fancy}
\rhead{}
\lhead{}

\setlength{\parindent}{0pt}
\setlength{\parskip}{1ex}

%%%%% Custom Commands
\newcommand{\calC}{\mathcal{C}}
\newcommand{\calE}{\mathcal{E}}
\newcommand{\calJ}{\mathcal{J}}
\newcommand{\calL}{\mathcal{L}}
\newcommand{\calT}{\mathcal{T}}
\newcommand{\calV}{\mathcal{V}}
	
\newcommand{\bbE}{\mathbb{E}}
\newcommand{\bbN}{\mathbb{N}}
\newcommand{\bbR}{\mathbb{R}}
\newcommand{\bbP}{\mathbb{P}}
\newcommand{\bbZ}{\mathbb{Z}}

\newcommand{\MCMC}{Markov Chain Monte Carlo}
\newcommand{\ihat}{\hat{\imath}}

%%%%% Algorithm Environment (In box)
%\newcounter{algo}[part]
%
%\newcommand{\algorithm}[2]{
%	\refstepcounter{algo}
%	\begin{tcolorbox}[breakable, enhanced, colback=white, arc=0mm]
%		{\bf Algorithm~\thealgo. #1} #2
%	\end{tcolorbox}
%}

\begin{document}
	\section{Notation}
		\begin{itemize}
			\item[$\calT$] Phylogeny with migration history
			\item[$d$] Number of distinct demes
			\item[$\Lambda$] Migration matrix $\Lambda = [\Lambda_{ij}]_{i,j=1}^d$ such that $\Lambda = \lambda_{i \rightarrow j}$ is the migration rate from deme $i$ into deme $j$ backwards in time. Note that self-migrations are forbidden, i.e. $\lambda_{ii} = 0$ $\forall i = 1,2, \dots, d$.
			\item[$\theta$] Effective population vector such that $\theta_i$ is the effective population in deme $i$
			\item[$g$] Mean generation length for individuals in the overall population ($g > 0$)
			\item[$M$] Total number of migration events
			\item[$n$] Number of leaves (tips) in the tree
		\end{itemize}
	
	\section{Structured Coalescent Likelihood}
		(Based on Ewing et al. (2004))
			\begin{equation}
				L_{n,d}(\calT) = \prod_{i=1}^d \prod_{\substack{j=1 \\ j \neq i}}^d \frac{1}{\theta_i^{c_i}} \lambda_{ij}^{m_{ij}} \prod_{r=2}^{2n + M -1} \exp \left\{ - \left( \frac{k_{i r} (k_{i r} - 1)}{2 \theta_i} + k_{i r} \lambda_{i j} \right)(t_{r-1} - t_r) \right\}
			\end{equation}
	
	\section{MCMC}
		Fix a structured coalescent phylogeny and the evolutionary parameters (migration matrix, effective populations). Construct a MCMC algorithm inspired by Ewing et al. (2004) with the following three types of proposal move
			\begin{enumerate}
				\item Migration Birth/Death Move
				\item Migration Pair Birth/Death Move
				\item Coalescent Node Split/Merge Move
			\end{enumerate}
		which obtain irreducibility over the space of migration histories on a fixed phylogeny with fixed evolutionary parameters.
		
		The acceptance probabilities of each of the above types of proposal move takes the form
			\begin{equation}
				\alpha(\calT' | \calT) = \min \left( 1, \frac{L(\calT') \pi(\calT') Q(\calT | \calT')}{L(\calT) \pi(\calT) Q(\calT' | \calT)} || \calJ || \right)
			\end{equation}
		where $L$ denotes the joint likelihood of the phylogeny and migration history, $\pi$ denotes the prior distribution on the phylogeny, $Q(\calT' | \calT)$ denotes the transition probability of obtaining $\calT'$ from $\calT$, and $\calJ$ denotes the Jacobian of the transformation.
		
		In all three cases, the transformations simply add or remove nodes from the migration history, and hence the term $||\calJ||$ is identically equal to 1. Similarly, the phylogeny is fixed and hence the prior ratio $\frac{\pi(\calT')}{\pi(\calT)} = 1$ also.
		
		\subsection{Migration Birth/Death Proposal}
			The migration birth/death proposal aims to add or remove a single migration node from the migration history, reassigning demes as necessary to maintain a consistent migration history. With probability $\frac{1}{2}$ a birth proposal is attempted, and otherwise a death proposal is attempted.
			
			\subsubsection{Ewing et al. (2004) Version}
				If a migration birth event is attempted, a new migration node $\ihat$ is proposed to be born at a location uniformly distributed along edge $\langle i, ip \rangle$. Otherwise, if a migration death event is attempted, set $\ihat$ to be the parent of $ip$ and remove node $ip$ from the edge $\langle i, \ihat \rangle$
				
				For any pair of vertices $i,j \in \calV$, let $\calT_{\langle i, j \rangle}$ denote the maximal subtree of $\calT$ containing edge $\langle i, j \rangle$ such that the terminal nodes of $\calT_{\langle i, j \rangle}$ are either migration or leaf nodes.
				
				Then the proposal is completed by selecting a deme $\hat{d}$ over the subtree $\calT_{\langle i, \ihat \rangle}$ such that $\hat{d}$ is not equal to the deme of any edge connected to $\calT_{\langle i, \ihat \rangle}$ or the deme of edge $\langle i, \ihat \rangle$ itself. If no deme $\hat{d}$ exists then the proposal is rejected with probability 1. Note that a migration birth/death move is not possible when there are only 2 demes as there is no way to choose a new deme consistent with the surrounding demes.
				
				The proposal ratio for the migration birth move is then given by
					\begin{align*}
						Q( \calT' | \calT) & = \bbP \left[ \substack{\text{Selected node} \\ \text{from} \, \calT} \right] \cdot \bbP[\substack{\text{Selected}\\ \text{deme}}] \cdot \bbP[\substack{\text{New node} \\ \text{time}}] \\
							& = \frac{1}{n-1+M} \cdot \frac{1}{c_b} \frac{1}{t_i - t_{ip}} \\
							& = \frac{1}{(n+M-1)c_b (t_i - t_{ip})}; \\[1ex]
						Q(\calT | \calT') & = \bbP[\substack{\text{Selected migration} \\ \text{node from} \, \calT'}] \cdot \bbP[\substack{\text{Selected} \\ \text{deme}}] \\
							& = \frac{1}{M+1} \cdot \frac{1}{c_d}; \\
						\frac{Q(\calT | \calT')}{Q(\calT' | \calT)} & = \frac{c_b (n+M-1) (t_i - t_{ip})}{c_d(M+1)}
					\end{align*}
				where $c_b$ and $c_d$ are the number of demes which could be proposed for the birth and death proposals respectively.
				
				Similarly, the proposal ratio for the migration death move is given by
					\[
						\frac{Q(\calT | \calT')}{Q(\calT' | \calT)} = \frac{c_d M}{c_b (n+M - 2) (t_i -t_{\hat{i}})}.
					\]
				
		\subsubsection{Updated Version (Uniform location on tree for birth, deterministic deme update for death)}
			If a migration birth proposal is attempted, a location is selected uniformly on the tree for a new migration node $\hat{i}$ to be born. Otherwise, a migration death proposal is attempted similarly to the migration death proposal of Ewing et al. The key difference for the migration death proposal is that the updated deme is selected deterministically to be the same as the deme above the selected migration event.
			
			This results in the modified proposal ratios
				\[
					\frac{Q(\calT | \calT')}{Q(\calT' | \calT)} = \frac{c_b \calL}{M+1}
				\]
			for the birth proposal, and
				\[
					\frac{Q(\calT | \calT')}{Q(\calT' | \calT)} = \frac{M}{c_b \calL}
				\]
			for the death proposal.
\end{document}